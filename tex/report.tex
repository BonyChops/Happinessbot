\documentclass[a4j]{jarticle}
\usepackage{float}
\usepackage[dvipdfmx]{graphicx}
%\usepackage{mediabb}
\makeatletter 
%https://qiita.com/ta_b0_/items/2619d5927492edbb5b03
\usepackage{listings,jlisting} %日本語のコメントアウトをする場合jlstlistingが必要
%ここからソースコードの表示に関する設定
\lstset{
  basicstyle={\ttfamily},
  identifierstyle={\small},
  commentstyle={\smallitshape},
  keywordstyle={\small\bfseries},
  ndkeywordstyle={\small},
  stringstyle={\small\ttfamily},
  frame={tb},
  breaklines=true,
  columns=[l]{fullflexible},
  numbers=left,
  xrightmargin=0zw,
  xleftmargin=3zw,
  numberstyle={\scriptsize},
  stepnumber=1,
  numbersep=1zw,
  lineskip=-0.5ex
}
%ここまでソースコードの表示に関する設定

\if0
---------------------------------こめこめこめこめこめこめこめこめこめこめこめ
\renewcommand{\thefigure}{\arabic{figure}}
\@addtoreset{figure}{section}
\makeatother
\makeatletter
\renewcommand{\thetable}{\arabic{table}}
\@addtoreset{table}{section}
\makeatother
\makeatletter
\renewcommand{\theequation}{\arabic{equation}}
\@addtoreset{equation}{section}
\makeatother
---------------------------------こめこめこめこめこめこめこめこめこめこめこめ
\fi
\usepackage{here}
\usepackage{amssymb}
\usepackage{amsmath}
\usepackage{url}
\usepackage{ascmac}
\usepackage{fancyvrb}
\usepackage{otf}



\title{工学実験実習II 実験報告書}
\author{幸せうんちくんbot}
\date{2020/02/16}



\newcommand{\unchi}[1]{\begin{screen}
    #1
\end{screen}}



\begin{document}

\maketitle
\section{目的}
幸せについて研究し、人間が思う幸せについてより深い理解をすること。

\section{手段}
SNSサービス「Twitter」を用いて実際につぶやかれたツイートをもとに幸せについて学ぶ。 \\
まず、定義済みの幸せワード(後述)で検索し、次に定義済みの不幸ワード(後述)が含まれていないかかを確認する \\
\quad 最後に、そのツイートの文字数を計測し、一定数以下であることを確認して、そのツイートを幸せツイートとして選択する。

\subsection{幸せワード}
ここでは幸せワードを以下と定義する。
\begin{itemize}
  \item テスト
  \item テスト
  \item テスト
\end{itemize}

\subsection{不幸ワード}
ここでは不幸を以下と定義する。なお、気分を害する可能性のあるワードも含まれているため、文字列の一部を伏せている。
\begin{itemize}
  \item テ*ト
  \item テ*ト
  \item テ*ト
\end{itemize}


\section{結果}
ここでは実際の結果を示す。
\subsection{幸せワードでの検索}
まず、幸せワードから検索するワードを選択する。今回は「うんち」を選択した。検索した結果は以下に示す通りである。
\begin{enumerate}
\item 
\begin{itembox}[l]{ツイート}
  はらへったなあ
\end{itembox}

\item 
\begin{itembox}[l]{2}
  はらへったなあ
\end{itembox}

\end{enumerate}
\section{結果}
以下に結果を示す.なお以下の結果は,PCvsPCの1戦を1手ずつすべて示したものである.\\



\section{考察}
この2ヶ月のCによるオセロゲーム製作によって,Cの基礎的な知識をつけることができた.

\end{document}